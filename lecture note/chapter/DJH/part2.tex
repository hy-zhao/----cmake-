\chapter{Introduction}
Algorithms + Data Structure = Programs\\
{Algorithms + Data Structure} $\times$ efficiency = Computation\\


\section{computational models}
\subsection{measure}
\begin{itemize}
\item 正确性
\item \stress{成本}
\begin{itemize}
\item 时间成本\ $T_A(n) = \max \left\{T_{0}(P)|| P \mid=n\right\}$ 
\item 空间成本: 除了输入之外,计算所要用的空间
\end{itemize}
\end{itemize}

\paragraph{Turning machine}
\includegraphics[scale=0.5]{TM}
\paragraph{Random Access Machine(RAM)}
\begin{itemize}
\item 寄存器顺序编号,总数没有限制
\item 每一个基本操作仅需常数时间(constant time)
\end{itemize}

\comment{TM 和 RAM都是对于一般计算模型的抽象和简化,可独立于具体的平台,对算法的效率进行可信的比较和评判。将运行时间转换成需要执行的基本操作次数}

\includegraphics[scale=0.5]{2-1}


\subsection{渐进分析}


\paragraph{big $\Omega$记号}
\begin{itemize}
\item $T(n)=\Omega(f(n)):$
\item $\exists c>0$, 当 $n\gg 2$ 后,有 $T(n) < c \cdot f(n)$
\end{itemize}

\paragraph{big $\Theta$记号}
\begin{itemize}
\item $T(n)=\Theta(f(n)):$
\item $\exists c_{1}>c_{2}>0$, 当 $n \gg 2$ 后, 有 $c_{1} \cdot f(n)>T(n)>c_{2} \cdot f(n)$
\item $\Theta$ 是 $\Omega$ 和 $O$ 的组合
\end{itemize}

\paragraph{大$O$ 记号(big-$O$ notation)}
\begin{itemize}
\item 随着计算规模的增长,计算成本考察的是增长趋势,而不考虑具体的操作次数和存储单元 \ra Asymptotic analysis
\begin{itemize}
\item $T(n)=O(f(n))$ iff $\exists\ c>0$, 当 $n \gg 2$ 后, 有 $\mathrm{T}(\mathrm{n})< \mathrm{f}(\mathrm{n})$
\item $\sqrt{5 n \cdot[3 n \cdot(n+2)+4]+6}< \sqrt{5 n \cdot\left[6 n^{2}+4\right]+6}<\sqrt{35 n^{3}+6}<6 \cdot n^{1.5}=O\left(n^{1.5}\right)$
\item 常系数可忽略
\item 低次项可忽略
\item upper-bond在n比较小的时候,不一定会在$T(n)$
\end{itemize}
\end{itemize}

\begin{enumerate}[label={(\arabic*)}]
\item constant function
\begin{itemize}
\item $2=2013=2013 \times 2013=O(1)$ 甚至 $2013^{2013}=O(1)$
\end{itemize}
\item 对数 $O(\log n)$
\begin{itemize}
\item 常底数无所谓: $\forall \mathrm{a}, \mathrm{b}>0, \log _{\mathrm{a}} n=\log _{\mathrm{a}} \mathrm{b} \cdot \log _{b} n=O\left(\log _{b} n\right)$
\item 常数次幂无所谓: $\forall c>0, \log n^{c}=c \cdot \log n=O(\log n)$
\item 对数多项式(poly-log function): $123 * \log^{321} n+\log ^{105}\left(n^{2}-n+1\right)=O\left(\log ^{321} n\right)$
\item 这类算法非常有效,复杂度无限接近与常数: $\forall\ c>0, \log n=O\left(n^{c}\right)$
\end{itemize}
\item 多项式(polynomial function)
\begin{itemize}
\item 一般地 : $a_{k} n^{k}+a_{k-1} n^{k-1}+\ldots+a_{1} n+a_{\theta}=O\left(n^{k}\right), a_{k}>0$
\item tractable
\item 线性(linear function): 所有 $O(\mathrm{n})$ 类函数
\end{itemize}
\item 指数(exponential function): $T(n)$
\begin{itemize}
\item $\forall c>1, \quad n^{c}=O\left(2^{n}\right)$
\item 这类算法的计算成本增长极快,通常认为不可忍受
\item intractable, incomputable
\item NP-complete: 不存在多项式时间内回答此问题的算法
\end{itemize}
\item 从 $O\left(\mathrm{n}^{c}\right)$ 到 $0\left(2^{\mathrm{n}}\right)$, 是从有效算法到无效算法的分水岭
\item 很多问题的 $O\left(2^{n}\right)$ 算法往往显而易见。然而,设计出 $O\left(\mathrm{n}^{c}\right)$ 算法却极其不易。甚至,有时注定地員能是徒劳无功
\end{enumerate}


\section{算法分析}
正确性(不变性$\times$ 单调性) + 复杂度

\subsection{C++指令类型}
\begin{itemize}
\item C++基本指令,RAM的基本指令
\item 分支转向: goto
\item 迭代循环: for(), while()
\item 调用 + 递归 \ra goto
\end{itemize}

\subsection{复杂度分析的主要方法}
\begin{itemize}
\item 迭代: 级数求和
\item 递归: 递归跟踪 + 递归方程
\item 猜测 + 验证
\end{itemize}

\paragraph{级数}
\begin{itemize}
\item 算数级数: 与末项平方同阶 \\ $\mathrm{T}(\mathrm{n})=1+2+\ldots+\mathrm{n}=\mathrm{n}(\mathrm{n}+1) / 2=O\left(\mathrm{n}^{2}\right)$
\item 幂方级数: 比幂次高出一阶 \\ $\sum_{\mathrm{k}=0}^{n} k^{d} \approx \int_{0}^{n} x^{d+1} d x=\left.\frac{1}{d+1} x^{d+1}\right|_{0} ^{n}=\frac{1}{d+1} n^{d+1}=O\left(n^{d+1}\right)$
\item 几何级数: 与末项同阶 \\ $T_{a}(n)=a^{e}+a^{1}+\cdots+a^{n}=\left(a^{n+1}-1\right) /(a-1) = O\left(a^{n}\right)$\\
$1+2+4+\ldots+2^{n}=2^{n+1}-1=O\left(2^{n+1}\right)=O\left(2^{n}\right)$
\item 收敛级数 \\$1 / 1 / 2+1 / 2 / 3+1 / 3 / 4+\ldots+1 /(n-1) / n=1-1 / n=O(1)$\\
$1+1 / 2^{2}+\ldots+1 / n^{2}<1+1 / 2^{2}+\ldots=\pi^{2 / 6}=O(1)$\\
$1 / 3+1 / 7+1 / 8+1 / 15+1 / 24+1 / 26+1 / 31+1 / 35+\ldots=O(1)$
\item 可能未必收敛,但是长度有限\\
调和级数: $h(n)=1+1 / 2+1 / 3+\ldots+1 / n=\Theta(\log n)$ \\ 
对数级数: $\log 1+\log 2+\log 3+\ldots+\log  n=\log (n !)=\Theta(n \log n)$
\end{itemize}

\paragraph{循环 vs 级数}
$1 \prec \log \log n \prec \log n \prec n^{\epsilon} \prec n^{c} \prec n^{\log n} \prec c^{n} \prec n^{n} \prec c^{c^{n}},\ 0<\epsilon<1<\mathrm{c}$.\\
\includegraphics[scale=0.5]{2-2}\\
\includegraphics[scale=0.5]{2-3}

\paragraph{Back-of-The-Envolope Calculation}

\begin{itemize}
\item 1 day $\approx 25 \times 4000=10^ 5 \mathrm{sec}$
\item 1 lifetime / century $\approx 100 \times 365 =3 \times 10^4 \times 10^5 =3 \times 10^9 \mathrm{sec}$
\item 三生三世 $\approx 300 \mathrm{yr}=10^ {10} \mathrm{sec}$
\end{itemize}

\paragraph{迭代与递归}
\includegraphics[scale=0.5]{2-4}\\
\includegraphics[scale=0.5]{2-5}\\
\includegraphics[scale=0.5]{2-6}\\
\includegraphics[scale=0.5]{2-7}\\
\comment{几何级数可知,其big O和最后一个数值的级数相同,而最后一个数就是数组的个数$n$}

\includegraphics[scale=0.5]{2-8}\\
\comment{普通的求解达到$O(2n)$, 上述情况最坏的情况降为$\frac{5}{3}$}

\section{Danymic programming}
fibonacci 的求和两种简便方法:
\begin{itemize}
\item memoization: 将已计算过示例的结果制表备查
\item Danymic programming: 
\begin{lstlisting}
    f = 0; g = 1;
    while(0<n--){
        g = g + f;
        f = g - f;
    }
    return g;
\end{lstlisting}
\end{itemize}

\begin{itemize}
    \item 公共子序列(subsequence):由序列中若干字符,按原相对次序构成
    \item 最长公共子序列(longest common subsequence), 可能有多个,但是长度是一定的
\end{itemize}


\subsection{LCS: 递归}
对于序列 $\mathrm{A}[0, \mathrm{n}]$ 和 $\mathrm{B}[0, \mathrm{~m}], \mathrm{LCS}(\mathrm{A}, \mathrm{B})$ 无非三种情况
\begin{enumerate}[label={(\arabic*)}]
\item 若n = -1或m = -1,则取作空序列 ("")
\item 若$ A[n]$='x'=$B[m]$, 则取作LCS $(A[\theta, n), B[\theta, m))$+ 'X' \LR decrease
\item 若$A[n] \neq B[m]$,则在两个字符中取一个稳定,另外一个单独的字符和稳定的字符比较,相同保留,不同去掉(两种取法,每一个都对应两种取法,可以发散出很多种) \LR devide
\item $O(2^n)$
\item 使用动态规划,可以消除其中的重复计算
\end{enumerate}














